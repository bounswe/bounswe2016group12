\hypertarget{dir_c6321af092c8aab3731ae43be8964c6b}{}\section{C\+:/\+Users/\+Ahmet Ercan Tekden/\+Documents/bounswe2016group12/\+W\+E\+B-\/\+B\+A\+C\+K\+E\+N\+D/\+Meanco/\+Meanco\+App/static/vis-\/4.17.0/examples/graph3d/playground Directory Reference}
\label{dir_c6321af092c8aab3731ae43be8964c6b}\index{C\+:/\+Users/\+Ahmet Ercan Tekden/\+Documents/bounswe2016group12/\+W\+E\+B-\/\+B\+A\+C\+K\+E\+N\+D/\+Meanco/\+Meanco\+App/static/vis-\/4.\+17.\+0/examples/graph3d/playground Directory Reference@{C\+:/\+Users/\+Ahmet Ercan Tekden/\+Documents/bounswe2016group12/\+W\+E\+B-\/\+B\+A\+C\+K\+E\+N\+D/\+Meanco/\+Meanco\+App/static/vis-\/4.\+17.\+0/examples/graph3d/playground Directory Reference}}
