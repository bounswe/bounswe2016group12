The company that developed vis.\+js for the main part, {\itshape almende} is ./we\+\_\+need\+\_\+help.md \char`\"{}not able to maintain the project at the moment\char`\"{}. So help from the community is very needed and welcome!

\subsection*{There are many ways to help\+:}

\subsubsection*{Answering questions}

There are new \href{//github.com/almende/vis/issues?q=is%3Aissue+is%3Aopen+label%3AQuestion+sort%3Acreated-desc}{\tt issues with questions} how to use vis.\+js opened almost every day. Be part of the community and help answer them!

A better way to ask questions on how to use vis.\+js is \href{https://stackoverflow.com/tags/vis.js}{\tt stackoverflow}. Questions are posed here also and need to be answered by the community. \href{https://stackoverflow.com/tags/vis.js}{\tt Please help answering questions} here also.

\subsubsection*{Closing old issues}

A new issue is often opened fast and then forgotten. Please help go trough \href{//github.com/almende/vis/issues?q=is%3Aissue+is%3Aopen+sort%3Acreated-asc}{\tt the old issues} (especially the \href{//github.com/almende/vis/issues?q=is%3Aissue+is%3Aopen+sort%3Acreated-asc+label%3AQuestion}{\tt questions}) and ask the creator of the issues if the problem still exists before closing the issue. The support team uses the {\bfseries issue inactive} label to mark these issues.

\subsubsection*{Improve the webpage}

The visjs.\+org webpage is hosted on the \href{//github.com/almende/vis/tree/gh-pages}{\tt gh-\/pages branch}. If you find a typo or anything else that should be improved feel free to create a pull-\/request to {\itshape gh-\/pages}. Please make changes in your own fork of gh-\/pages so the support team can view the changes in your hosted fork.

\subsubsection*{Create new examples}

We have \href{//github.com/almende/vis/tree/develop/examples}{\tt a collection of examples}. Please help by creating interesting new ones that show a specific problem or layout. Keep the examples easy to understand for beginners and remove unnecessary clutter.

\subsubsection*{Provide interesting showcases}

If you use vis.\+js to develop something beautiful feel free to create a pull-\/request to our show cases page in the gh-\/pages branch\mbox{]}(//github.com/almende/vis/tree/gh-\/pages/showcase). \href{http://visjs.org/showcase/index.html}{\tt These showcases are displayed on our webpage} and we are always looking for new examples.

\subsubsection*{Confirming and fixing bugs}

Every software has bugs. We also have \href{https://github.com/almende/vis/issues?q=is%3Aissue+is%3Aopen+label%3ABug+sort%3Areactions-%2B1-desc}{\tt quite a nice collection} ;-\/) Feel free to fix as many bugs as you want!

You can not only help by fixing bugs, but also by confirming the bug or even creating a minimal code example to prove this bug exists.

\subsubsection*{Implementing Feature-\/\+Requests}

A lot of people have a lot of ideas for improving vis.\+js. \href{https://github.com/almende/vis/labels/Feature-Request}{\tt We label these issues as {\bfseries Feature-\/\+Request}}. Feel free to implement a new feature by creating a new Pull-\/\+Request.

\href{//github.com/almende/vis/issues?q=is%3Aissue+is%3Aopen+label%3A%22For+everyone%21%22+sort%3Areactions-%2B1-desc}{\tt Some issues are labeled {\bfseries For everybody!}}. These are a good starting point.

\subsubsection*{Reviewing Pull-\/\+Requests}

We use \href{//help.github.com/articles/about-pull-request-reviews/}{\tt Git\+Hub\textquotesingle{}s two-\/step review} to make sure pull-\/requests are clean. You can help by checking out pull-\/request branches and testing them. You also can comment on lines of code and make sure the pull-\/request introduces no new bugs or typos.

\subsection*{Creating Pull Requests}

There are some rules for pull-\/request\+:


\begin{DoxyItemize}
\item All pull-\/request must be to the \href{//github.com/almende/vis/tree/develop}{\tt develop-\/branch}. Pull-\/request against the \href{//github.com/almende/vis/tree/master}{\tt master-\/branch} must be closed. (Changes to \href{//github.com/almende/vis/tree/gh-pages}{\tt gh-\/pages} are also ok.)
\item Only commit changes done in the source files in the folder {\ttfamily lib}, not to the builds which are located in the folder {\ttfamily dist}.
\item Keep your changes small and clear. Only work on one topic at one time and only change lines of code that you have to change to reach your goal.
\item Test your changes before creating a pull-\/request. The easiest way is to open the existing examples and playing with them.
\item If you are fixing or implementing an existing issue, please refer to it in the description and in the commit message.
\item If you are introducing a new feature, add some documentation and a new example to make it easy to adapt.
\item If you introduce breaking changes, like changing the signature of a public function, point that out in your description. Breaking changes result in a new major release.
\item Always adapt to the code style of the existing source. Never adapt existing code to your personal taste. \+:trollface\+:
\item Pull-\/requests must be reviewed by at least two member of the support team. The First must approve the pull-\/request, the second can than merge after also checking it.
\end{DoxyItemize}

{\bfseries Happy Helping!!} 