\href{https://gitter.im/vis-js/Lobby?utm_source=badge&utm_medium=badge&utm_campaign=pr-badge&utm_content=badge}{\tt }

\href{https://github.com/almende/vis/issues/1781}{\tt }

Vis.\+js is a dynamic, browser based visualization library. The library is designed to be easy to use, handle large amounts of dynamic data, and enable manipulation of the data. The library consists of the following components\+:


\begin{DoxyItemize}
\item Data\+Set and Data\+View. A flexible key/value based data set. Add, update, and remove items. Subscribe on changes in the data set. A Data\+Set can filter and order items, and convert fields of items.
\item Data\+View. A filtered and/or formatted view on a Data\+Set.
\item Graph2d. Plot data on a timeline with lines or barcharts.
\item Graph3d. Display data in a three dimensional graph.
\item Network. Display a network (force directed graph) with nodes and edges.
\item Timeline. Display different types of data on a timeline.
\end{DoxyItemize}

The vis.\+js library is developed by \href{http://almende.com}{\tt Almende B.\+V}.

\subsection*{Install}

Install via npm\+: \begin{DoxyVerb}$ npm install vis
\end{DoxyVerb}


Install via bower\+: \begin{DoxyVerb}$ bower install vis
\end{DoxyVerb}


Link via cdnjs\+: \href{http://cdnjs.com}{\tt http\+://cdnjs.\+com}

Or download the library from the github project\+: \href{https://github.com/almende/vis.git}{\tt https\+://github.\+com/almende/vis.\+git}.

\subsection*{Load}

To use a component, include the javascript and css files of vis in your web page\+:


\begin{DoxyCode}
<!DOCTYPE HTML>
<html>
<head>
  <script src="components/vis/dist/vis.js"></script>
  <link href="components/vis/dist/vis.css" rel="stylesheet" type="text/css" />
</head>
<body>
  <script type="text/javascript">
    // ... load a visualization
  </script>
</body>
</html>
\end{DoxyCode}


or load vis.\+js using require.\+js. Note that vis.\+css must be loaded too.


\begin{DoxyCode}
require.config(\{
  paths: \{
    vis: 'path/to/vis/dist',
  \}
\});
require(['vis'], function (math) \{
  // ... load a visualization
\});
\end{DoxyCode}


A timeline can be instantiated as\+:


\begin{DoxyCode}
var timeline = new vis.Timeline(container, data, options);
\end{DoxyCode}


Where {\ttfamily container} is an H\+T\+ML element, {\ttfamily data} is an Array with data or a Data\+Set, and {\ttfamily options} is an optional object with configuration options for the component.

\subsection*{Example}

A basic example on loading a Timeline is shown below. More examples can be found in the \href{https://github.com/almende/vis/tree/master/examples}{\tt examples directory} of the project.


\begin{DoxyCode}
<!DOCTYPE HTML>
<html>
<head>
  <title>Timeline basic demo</title>
  <script src="vis/dist/vis.js"></script>
  <link href="vis/dist/vis.css" rel="stylesheet" type="text/css" />

  <style type="text/css">
    body, html \{
      font-family: sans-serif;
    \}
  </style>
</head>
<body>
<div id="visualization"></div>

<script type="text/javascript">
  var container = document.getElementById('visualization');
  var data = [
    \{id: 1, content: 'item 1', start: '2013-04-20'\},
    \{id: 2, content: 'item 2', start: '2013-04-14'\},
    \{id: 3, content: 'item 3', start: '2013-04-18'\},
    \{id: 4, content: 'item 4', start: '2013-04-16', end: '2013-04-19'\},
    \{id: 5, content: 'item 5', start: '2013-04-25'\},
    \{id: 6, content: 'item 6', start: '2013-04-27'\}
  ];
  var options = \{\};
  var timeline = new vis.Timeline(container, data, options);
</script>
</body>
</html>
\end{DoxyCode}


\subsection*{Build}

To build the library from source, clone the project from github \begin{DoxyVerb}$ git clone git://github.com/almende/vis.git
\end{DoxyVerb}


The source code uses the module style of node (require and module.\+exports) to organize dependencies. To install all dependencies and build the library, run {\ttfamily npm install} in the root of the project. \begin{DoxyVerb}$ cd vis
$ npm install
\end{DoxyVerb}


Then, the project can be build running\+: \begin{DoxyVerb}$ npm run build
\end{DoxyVerb}


To automatically rebuild on changes in the source files, once can use \begin{DoxyVerb}$ npm run watch
\end{DoxyVerb}


This will both build and minify the library on changes. Minifying is relatively slow, so when only the non-\/minified library is needed, one can use the {\ttfamily watch-\/dev} script instead\+: \begin{DoxyVerb}$ npm run watch-dev
\end{DoxyVerb}


\subsection*{Custom builds}

The folder {\ttfamily dist} contains bundled versions of vis.\+js for direct use in the browser. These bundles contain all the visualizations and include external dependencies such as hammer.\+js and moment.\+js.

The source code of vis.\+js consists of commonjs modules, which makes it possible to create custom bundles using tools like \href{http://browserify.org/}{\tt Browserify} or \href{http://webpack.github.io/}{\tt Webpack}. This can be bundling just one visualization like the Timeline, or bundling vis.\+js as part of your own browserified web application.

{\itshape Note that hammer.\+js version 2 is required as of v4.}

\paragraph*{Prerequisites}

Before you can do a build\+:


\begin{DoxyItemize}
\item Install node.\+js and npm on your system\+: \href{https://nodejs.org/}{\tt https\+://nodejs.\+org/}
\item Install the following modules using npm\+: {\ttfamily browserify}, {\ttfamily babelify}, and {\ttfamily uglify-\/js}\+:
\end{DoxyItemize}


\begin{DoxyCode}
$ [sudo] npm install -g browserify babelify uglify-js
\end{DoxyCode}



\begin{DoxyItemize}
\item Download or clone the vis.\+js project\+:
\end{DoxyItemize}


\begin{DoxyCode}
$ git clone https://github.com/almende/vis.git
\end{DoxyCode}



\begin{DoxyItemize}
\item Install the dependencies of vis.\+js by running {\ttfamily npm install} in the root of the project\+:
\end{DoxyItemize}


\begin{DoxyCode}
$ cd vis
$ npm install
\end{DoxyCode}


\paragraph*{Example 1\+: Bundle a single visualization}

For example, to create a bundle with just the Timeline and Data\+Set, create an index file named {\bfseries custom.\+js} in the root of the project, containing\+:


\begin{DoxyCode}
exports.DataSet = require('./lib/DataSet');
exports.Timeline = require('./lib/timeline/Timeline');
\end{DoxyCode}


Then create a custom bundle using browserify, like\+: \begin{DoxyVerb}$ browserify custom.js -t babelify -o vis-custom.js -s vis
\end{DoxyVerb}


This will generate a custom bundle {\itshape vis-\/custom.\+js}, which exposes the namespace {\ttfamily vis} containing only {\ttfamily Data\+Set} and {\ttfamily Timeline}. The generated bundle can be minified using uglifyjs\+: \begin{DoxyVerb}$ uglifyjs vis-custom.js -o vis-custom.min.js
\end{DoxyVerb}


The custom bundle can now be loaded like\+:


\begin{DoxyCode}
<!DOCTYPE HTML>
<html>
<head>
  <script src="vis-custom.min.js"></script>
  <link href="dist/vis.min.css" rel="stylesheet" type="text/css" />
</head>
<body>
  ...
</body>
</html>
\end{DoxyCode}


\paragraph*{Example 2\+: Exclude external libraries}

The default bundle {\ttfamily vis.\+js} is standalone and includes external dependencies such as hammer.\+js and moment.\+js. When these libraries are already loaded by the application, vis.\+js does not need to include these dependencies itself too. To build a custom bundle of vis.\+js excluding moment.\+js and hammer.\+js, run browserify in the root of the project\+: \begin{DoxyVerb}$ browserify index.js -t babelify -o vis-custom.js -s vis -x moment -x hammerjs
\end{DoxyVerb}


This will generate a custom bundle {\itshape vis-\/custom.\+js}, which exposes the namespace {\ttfamily vis}, and has moment and hammerjs excluded. The generated bundle can be minified with uglifyjs\+: \begin{DoxyVerb}$ uglifyjs vis-custom.js -o vis-custom.min.js
\end{DoxyVerb}


The custom bundle can now be loaded as\+:


\begin{DoxyCode}
<!DOCTYPE HTML>
<html>
<head>
  
  <script src="http://cdnjs.cloudflare.com/ajax/libs/moment.js/2.7.0/moment.min.js"></script>
  <script src="http://cdnjs.cloudflare.com/ajax/libs/hammer.js/1.1.3/hammer.min.js"></script>

  
  <script src="vis-custom.min.js"></script>
  <link href="dist/vis.min.css" rel="stylesheet" type="text/css" />
</head>
<body>
  ...
</body>
</html>
\end{DoxyCode}


\paragraph*{Example 3\+: Bundle vis.\+js as part of your (commonjs) application}

When writing a web application with commonjs modules, vis.\+js can be packaged automatically into the application. Create a file {\bfseries app.\+js} containing\+:


\begin{DoxyCode}
var moment = require('moment');
var DataSet = require('vis/lib/DataSet');
var Timeline = require('vis/lib/timeline/Timeline');

var container = document.getElementById('visualization');
var data = new DataSet([
  \{id: 1, content: 'item 1', start: moment('2013-04-20')\},
  \{id: 2, content: 'item 2', start: moment('2013-04-14')\},
  \{id: 3, content: 'item 3', start: moment('2013-04-18')\},
  \{id: 4, content: 'item 4', start: moment('2013-04-16'), end: moment('2013-04-19')\},
  \{id: 5, content: 'item 5', start: moment('2013-04-25')\},
  \{id: 6, content: 'item 6', start: moment('2013-04-27')\}
]);
var options = \{\};
var timeline = new Timeline(container, data, options);
\end{DoxyCode}


Install the application dependencies via npm\+: \begin{DoxyVerb}$ npm install vis moment
\end{DoxyVerb}


The application can be bundled and minified\+: \begin{DoxyVerb}$ browserify app.js -o app-bundle.js -t babelify 
$ uglifyjs app-bundle.js -o app-bundle.min.js
\end{DoxyVerb}


And loaded into a webpage\+:


\begin{DoxyCode}
<!DOCTYPE HTML>
<html>
<head>
  <link href="node\_modules/vis/dist/vis.min.css" rel="stylesheet" type="text/css" />
</head>
<body>
  <div id="visualization"></div>

  <script src="app-bundle.min.js"></script>
</body>
</html>
\end{DoxyCode}


\subsection*{Test}

To test the library, install the project dependencies once\+: \begin{DoxyVerb}$ npm install
\end{DoxyVerb}


Then run the tests\+: \begin{DoxyVerb}$ npm test
\end{DoxyVerb}


\subsection*{License}

Copyright (C) 2010-\/2015 Almende B.\+V.

Vis.\+js is dual licensed under both


\begin{DoxyItemize}
\item The Apache 2.\+0 License \href{http://www.apache.org/licenses/LICENSE-2.0}{\tt http\+://www.\+apache.\+org/licenses/\+L\+I\+C\+E\+N\+S\+E-\/2.\+0}
\end{DoxyItemize}

and


\begin{DoxyItemize}
\item The M\+IT License \href{http://opensource.org/licenses/MIT}{\tt http\+://opensource.\+org/licenses/\+M\+IT}
\end{DoxyItemize}

Vis.\+js may be distributed under either license. 